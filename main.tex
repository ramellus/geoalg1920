\documentclass[twoside,openright,titlepage,numbers=noenddot,%1headlines,
               headinclude,footinclude,cleardoublepage=empty,abstract=on,
               BCOR=23mm,paper=letter,fontsize=11pt
               ]{scrreprt}
\input{classicthesis-config}
\usepackage{amsmath,
amsfonts,
amssymb,
amsthm,
epstopdf,
epigraph,
yfonts,
xcolor,
graphicx,
dsfont,
newlfont,
listings,
color,
booktabs,	
subfiles,
lmodern,
titlesec,
tikz,
thmtools,
stmaryrd,
lipsum,
textcase,
setspace,
hyperref,
marginnote}
\usepackage{nicefrac}
\usepackage{EuScript}
\usepackage{fourier, palatino}
\newtheoremstyle{custom-up}% name of the style to be used
{1pt}% measure of space to leave above the theorem. E.g.: 3pt
{1pt}% measure of space to leave below the theorem. E.g.: 3pt
{}% name of font to use in the body of the theorem
{}% measure of space to indent
{}% name of head font
{}% punctuation between head and body
{2pt}% space after theorem head; " " = normal interword space
{\llap{{\small\textcolor{black}{\textbf{\thmnumber{#2}}}}\hskip2mm}\textbf{\thmname{#1}}\hskip1mm\thmnote{(#3)}}%
\theoremstyle{custom-up}
\newtheorem*{axiom}{Assioma}
\newtheorem*{question}{Domanda}
\newtheorem*{exercise}{Esercizio}
\newtheorem{definition}{Definizione}[section]
\newtheorem{theorem}[definition]{Teorema}
\newtheorem{corollary}[definition]{Corollario} 
\newtheorem{lemma}[definition]{Lemma} 
\newtheorem{proposition}[definition]{Proposizione}
\newtheorem{example}[definition]{Esempio}
\newtheorem{remark}[definition]{Osservazione}
\newtheorem{claim}[definition]{Claim}
\newtheorem*{theorem*}{Theorem}

	\newcommand{\bb}[1]{\mathds{#1}}
	\newcommand{\fk}[1]{\mathfrak{#1}}
	\newcommand{\sff}[1]{\mathsf{#1}}
	\newcommand{\cl}[1]{\EuScript{#1}}
	\newcommand{\comp}[1]{#1^{\mathsf{c}}}
	\renewcommand{\P}{\cl{P}}
	\newcommand{\finset}[1]{\cl{P}_{\mathsf{fin}}(#1)}
	\newcommand{\struct}[1]{\mathds{#1}} % use this for "famous" sets and the like
	\newcommand{\sm}{\smallsetminus}
	\newcommand{\quot}[2]{{#1}/{#2}}
	\newcommand{\wt}[1]{\widetilde{#1}}
	\newcommand{\into}{\hookrightarrow}
	\newcommand{\then}{\rightarrow}
	\newcommand{\meet}{\land}
	\newcommand{\join}{\lor}
	\newcommand{\onto}{\twoheadrightarrow}
	\newcommand{\NN}{\struct{N}}
	\newcommand{\QQ}{\struct{Q}}
	\newcommand{\ZZ}{\struct{Z}}
	\newcommand{\RR}{\struct{R}}
	\newcommand{\PP}{\struct{P}}
	\newcommand{\CC}{\struct{C}}
	\renewcommand{\AA}{\struct{A}}
	\newcommand{\KK}{\struct{K}}
	\newcommand{\Q}[1]{\struct{Q}_{#1}}
	\newcommand{\Z}[1]{\struct{Z}_{#1}}
	\renewcommand{\SS}{\struct{S}}
	\newcommand{\U}{\cl{U}}
	\newcommand{\V}{\cl{V}}
	\newcommand{\W}{\cl{W}}
	\newcommand{\A}{\cl{A}}
	\newcommand{\M}{\cl{M}}
	\newcommand{\N}{\cl{N}}
	\newcommand{\B}{\cl{B}}
	\newcommand{\X}{\cl{X}}
	\newcommand{\Y}{\cl{Y}}
	\renewcommand{\Z}{\cl{Z}}
	\newcommand{\G}{\cl{G}}
	\newcommand{\F}{\mathcal{F}}
	\newcommand{\Gg}{\mathfrak{G}}
	\renewcommand{\O}{\cl{O}}
	\renewcommand{\L}{\cl{L}}
	\renewcommand{\emph}[1]{\textbf{#1}}
	\renewcommand{\geq}{\geqslant}
	\renewcommand{\leq}{\leqslant}
	\renewcommand{\epsilon}{\varepsilon}
	\newcommand{\interior}[1]{{#1}^{\circ}}
	\newcommand{\satisfies}{\vDash}
	\newcommand{\follows}{\vdash}
	\newcommand{\de}{\partial}
	\newcommand{\cinf}{\cl{C}^{\infty}}
	\newcommand{\Man}{\textsf{Man}}
	\newcommand{\isom}{\xrightarrow{\cong}}
	\newcommand{\define}{\hskip1mm\dot{=}\hskip1mm}
	\renewcommand{\o}{\mathbf{0}}
	\newcommand{\MK}{\mathsf{MK}}
	\newcommand{\Ord}{\mathrm{Ord}}
	\newcommand{\Card}{\mathrm{Card}}
	\newcommand{\ot}{\mathrm{ot}}
	\newcommand{\ran}{\mathrm{ran}}
	\newcommand{\dom}{\mathrm{dom}}
	\newcommand{\rank}{\mathrm{rank}}
	\newcommand{\AC}{\mathrm{AC}}
	\newcommand{\cof}{\mathrm{cof}}
	\newcommand{\ar}{\mathrm{ar}}
	\newcommand{\ClF}{\mathrm{Cl}_{\F}}
	\newcommand{\C}{\mathbf{C}}
\title{Geometria Algebrica 1920}
\author{Simone Ramello}
\date{\today} 
\begin{document}
\maketitle
\chapter{Irriducibilità e dimensione}
\section{Richiami da Istituzioni}
Fissiamo un campo $k$ algebricamente chiuso. Con il termine ``varietà'' indicheremo una \textbf{varietà quasi proiettiva}, ovvero un sottospazio localmente chiuso\footnote{Vale a dire, un sottospazio aperto nella propria chiusura.} di $\PP^n$. Estendiamo un po' la nomenclatura: diremo che una varietà è \textbf{affine} se è isomorfa ad un chiuso $X \subseteq \AA^n$. Esempi canonici sono tutti i chiusi affini e, in maniera meno ovvia, ogni aperto principale di una varietà affine. Diremo \textbf{aperto affine} per indicare un aperto di una varietà che, visto a sua volta come varietà, è affine.
\begin{remark}
    Gli aperti affini costituiscono una base per la topologia di Zariski di una varietà $X$. Infatti, posso decomporre $X$ lungo le \textbf{carte affini}
    \[ U_i := \{(x_0: \dots :x_n) \in \PP^n: x_i \neq 0\} \]
    che ricoprono $\PP^n$ nel modo seguente,
    \[ X = \bigcup_{i=0}^{n} X\cap U_i \]
    e indicate con $X_i := X \cap U_i$, questi ultimi sono localmente chiusi in $U_i$. Ciascuno di essi può dunque essere scritto come unione di aperti principali (che sappiamo costituire una base per la topologia di ciascuna varietà), che sono a loro volta aperti affini.
\end{remark}
\begin{exercise}
    Mostrare che
    \begin{enumerate}
        \item un chiuso di una varietà affine è una varietà affine,
        \item il prodotto di varietà affini è una varietà affine,
        \item l'intersezione di aperti affini è un aperto affine.
    \end{enumerate}
\end{exercise}
\subsection{Funzioni regolari e morfismi}
\begin{definition}
    Se $X$ è una varietà e $f: X \to k$, diremo che $f$ è \textbf{regolare} se è localmente quoziente di due polinomi omogenei del medesimo grado a denominatore non-nullo; se $X \subseteq U_i$ per qualche $i$, in particolare, sarà regolare se è localmente quoziente di polinomi a denominatore non-nullo.
\end{definition}
\begin{definition}
    Indichiamo con $\O(X)$ la $k$-algebra delle funzioni regolari su $X$. Nel caso in cui $X$ sia affine, scriveremo anche $k[X]$.
\end{definition}
\begin{remark}
    Se $X$ è affine, si ha $\O(X) \cong \frac{k[x_1, \dots x_n]}{I(X)}$, dove $I(X)$ è l'ideale dei polinomi che si annullano su $X$. Se $X$ è proiettiva e connessa, $\O(X) \cong k$.
\end{remark}
\subsection{Irriducibilità}
\begin{definition}
    Uno spazio topologico $X$ si dice \textbf{irriducibile} se \textit{non} esistono due chiusi $C_1, C_2 \subsetneq X$ non-vuoti tali che $X = C_1 \cup C_2$.
\end{definition}
\begin{exercise}
    Se $X$ è uno spazio topologico non vuoto sono equivalenti
    \begin{enumerate}
        \item $X$ irriducibile,
        \item ogni coppia di aperti non vuoti ha intersezione non vuota,
        \item ogni aperto non vuoto è denso in $X$.
    \end{enumerate}
    Inoltre, se $Y \subseteq X$ è denso, allora $X$ irriducibile $\iff Y$ irriducibile.
\end{exercise}
\subsection{Conseguenze del Nullstellensatz}
Ricordiamo che il Nullstellensatz fornisce una biezione fra i chiusi di una varietà affine $X$ e gli ideali radicali di $k[x_1, \dots x_n]$ che contengono $I(X)$. Questa biezione si compone con la proiezione al quoziente fornendo una biezione fra i chiusi di $X$ e gli ideali radicali di $k[X]$. L'ultima biezione si può anche scrivere direttamente: se $Y \subseteq X$, indichiamo con $I_{X}(Y) = \{f \in k[X]: f\vert_{Y} \equiv 0\}$.
\begin{center}
    Finire richiami!
\end{center}
\section{Dimensione topologica}
Sia $X$ uno spazio topologico; tenendo a mente il modello degli spazi noetheriani, indichiamo con
\begin{align*} 
    \dim(X) := \sup \{n \in \NN:& \ \text{esiste una catena di chiusi non vuoti irriducibili}\\ \ &Z_0 \subsetneq Z_1 \subsetneq Z_2 \subsetneq \dots \subsetneq Z_n \subseteq X\}
\end{align*}
la \textbf{dimensione (topologica) di $X$}.
\begin{example}
    Se $\# X = 1$, $\dim(X) = 0$. Similmente, siccome in $\AA^1$ gli unici chiusi irriducibili sono $\AA^1$ e i punti, le catene massimali hanno tutte la forma
    \[ \{\star\} \subseteq \AA^1, \]
    da cui $\dim(\AA^1) = 1$. In generale, mostrare che $\AA^n$ e $\PP^n$ hanno dimensione $n$ è molto più complicato, e ci vorrà un po' di lavoro.
\end{example}
\begin{remark}
    Contrariamente all'intuizione, non tutti gli spazi noetheriani hanno dimensione finita. Se ad esempio si considera $[0,1]$ con i chiusi della forma $Z_n := [-\frac{1}{n}, 1]$, questo spazio risulta noetheriano (soddisfa la condizione catenaria \textit{discendente}) ma ha dimensione infinita (non soddisfa quella \textit{ascendente}).
\end{remark}
\begin{proposition}
    Sia $X$ uno spazio topologico, allora:
    \begin{enumerate}
        \item $Y \subseteq X$ implica che $\dim(Y) \leq \dim(X)$,
        \item se $X$ è noetheriano e $X = X_1 \cup \dots \cup X_r$ è la sua decomposizione in irriducibili, allora
        \[ \dim(X) = \max_{i\leq r}\dim(X_i), \]
        \item se $X$ è irriducibile e ha dimensione finita, allora $Y \subsetneq X$ implica $\dim(Y) < \dim(X)$,
        \item se $X$ è noetheriano di dimensione finita e $Y \subseteq X$ è chiuso e ha la stessa dimensione dello spazio ambiente, allora $Y$ contiene una componente irriducibile di dimensione $\dim(X)$,
        \item se $\{U_\alpha\}_{\alpha \in A}$ è un ricoprimento aperto di $X$,
        \[ \dim(X) = \sup_{\alpha \in A} \dim(U_\alpha). \]
    \end{enumerate}
\end{proposition}
\begin{proof}
    Mostriamo $(1)$, gli altri sono esercizi. Siano
    \[ Z_0 \subsetneq Z_1 \subsetneq Z_2 \subsetneq \dots \subsetneq Z_n \subseteq Y \]
    chiusi irriducibili non vuoti, allora\footnote{Ovviamente la chiusura è in $X$.}
    \[ \overline{Z_0} \subsetneq \overline{Z_1} \subsetneq \dots \subsetneq \overline{Z_n} \subseteq X \]
    sono chiusi irriducibili non vuoti di $X$. Passando al limite, $\dim(X) \geq \dim(Y)$.
\end{proof}
\begin{definition}
    Diciamo che $X$ è \textbf{equidimensionale}, o che ha dimensione pura, se ogni componente irriducibile ha la stessa dimensione.
\end{definition}
\begin{remark}
    Se $X$ ha dimensione $n$, come testimoniato da
    \[ Z_0 \subsetneq Z_1 \subsetneq \dots \subsetneq Z_n \subseteq X, \]
    allora $\dim(Z_i) = i$ per ogni $i = 0, \dots n$. Ne consegue che $X$ è irriducibile se e solo se $X = Z_n$ (altrimenti potrei allungare la catena). Si noti che se $X$ è almeno $T_1$ (come nel caso di Zariski), $\#Z_0 = 1$ (perché altrimenti potrei allungare la catena).
\end{remark}
\begin{example}
    In generale non è vero\footnote{Lo sarà per varietà.} che la dimensione di un aperto denso corrisponda a quella dello spazio ambiente; se ad esempio si considera $[0,1]$ con la topologia $\{\emptyset, X, \{1\}\}$, allora risulta avere dimensione $1$ ma $\dim \{1\} = 0$, nonostante quest'ultimo sia denso.
\end{example}
Quando parleremo di \textbf{dimensione di una varietà} intenderemo sempre la sua dimensione topologica con la topologia di Zariski.
\end{document}